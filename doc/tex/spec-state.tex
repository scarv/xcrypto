\subsection{\XCRYPTO state}
\label{sec:spec:state}

% =============================================================================

\begin{table}[p]
\begin{center}
\begin{tabular}{|c|c|c|c|l|l|}
\hline
  \multicolumn{3}{|c|}{CSR Address}    & Hex & Use and Accessibility & Section \\
  \multicolumn{1}{|c }{$11 \RANGE 10$} 
& \multicolumn{1}{|c }{$ 9 \RANGE  8$}
& \multicolumn{1}{|c|}{$ 7 \RANGE  4$} &     &                       &         \\
\hline
\VERB{10} & \VERB{00} & \VERB{0000} & $\RADIX{800}{16}$ & $\SPR{uxcrypto}$ & \REFSEC{sec:spec:state:1}   \\
\VERB{10} & \VERB{00} & \VERB{0001} & $\RADIX{801}{16}$ & $\SPR{ubop}    $ & \REFSEC{sec:spec:state:2:3} \\
\VERB{10} & \VERB{00} & \VERB{0010} & $\RADIX{802}{16}$ & $\SPR{ufenl}   $ & \REFSEC{sec:spec:state:2:6} \\
\hline
\end{tabular}
\end{center}
\caption{An overview of \XCRYPTO-related CSRs.}
\label{tab:csr}
\end{table}

% -----------------------------------------------------------------------------

\begin{figure}[p]
\begin{center}
\begin{bytefield}[bitwidth={1.4em},bitheight={12.0ex},endianness=big]{32}
\bitheader{0-31}               
\\
  \bitbox{16}{\color{black}    \rule{\width}{\height}}
& \bitbox{ 1}{\rotatebox{90}{\small$\ID{ALZETTE}  $}}
& \bitbox{ 1}{\rotatebox{90}{\small$\ID{CHACHA20} $}}
& \bitbox{ 1}{\rotatebox{90}{\small$\ID{SM4}      $}}
& \bitbox{ 1}{\rotatebox{90}{\small$\ID{SHA3}     $}}
& \bitbox{ 1}{\rotatebox{90}{\small$\ID{SHA2}     $}}
& \bitbox{ 1}{\rotatebox{90}{\small$\ID{AES }     $}}
& \bitbox{ 1}{\rotatebox{90}{\small$\ID{MASK}     $}}
& \bitbox{ 1}{\rotatebox{90}{\small$\ID{LEAK}     $}}
& \bitbox{ 1}{\rotatebox{90}{\small$\ID{MP  }     $}}
& \bitbox{ 1}{\rotatebox{90}{\small$\ID{PACK}_{16}$}}
& \bitbox{ 1}{\rotatebox{90}{\small$\ID{PACK}_{ 8}$}}
& \bitbox{ 1}{\rotatebox{90}{\small$\ID{PACK}_{ 4}$}}
& \bitbox{ 1}{\rotatebox{90}{\small$\ID{PACK}_{ 2}$}}
& \bitbox{ 1}{\rotatebox{90}{\small$\ID{BIT }     $}}
& \bitbox{ 1}{\rotatebox{90}{\small$\ID{MEM }     $}}
& \bitbox{ 1}{\rotatebox{90}{\small$\ID{RND }     $}}
\end{bytefield}
\end{center}
\caption{A diagrammatic description of the $\SPR{uxcrypto}$ register.}
\label{fig:csr:uxcrypto}
\end{figure}

\begin{figure}[p]
\begin{center}
\begin{bytefield}[bitwidth={1.4em},bitheight={12.0ex},endianness=big]{32}
\bitheader{0-31}               
\\
& \bitbox{16}{\color{black}    \rule{\width}{\height}}
  \bitbox{ 8}{\rotatebox{90}{\small$\ID{BOP}_{ 1} $}}
& \bitbox{ 8}{\rotatebox{90}{\small$\ID{BOP}_{ 0} $}}
\end{bytefield}
\end{center}
\caption{A diagrammatic description of the $\SPR{ubop}$     register.}
\label{fig:csr:ubop}
\end{figure}

\begin{figure}[p]
\begin{center}
\begin{bytefield}[bitwidth={1.4em},bitheight={12.0ex},endianness=big]{32}
\bitheader{0-31}               
\\
& \bitbox{32}{\color{lightgray}\rule{\width}{\height}}
\end{bytefield}
\end{center}
\caption{A diagrammatic description of the $\SPR{ufenl}$    register.}
\label{fig:csr:ufenl}
\end{figure}

% -----------------------------------------------------------------------------

\begin{table}[p]
\begin{center}
\begin{tabular}{|l|c|c|l|}
\hline
Field            & Index           & Access & Description                                                            \\ 
\hline
$\ID{ALZETTE}  $ & $           15$ & R      & Is the Alzette         class of instructions                supported? \\
$\ID{CHACHA20} $ & $           14$ & R      & Is the ChaCha20        class of instructions                supported? \\
$\ID{SM4}      $ & $           13$ & R      & Is the SM4             class of instructions                supported? \\
$\ID{SHA3}     $ & $           12$ & R      & Is the SHA3            class of instructions                supported? \\
$\ID{SHA2}     $ & $           11$ & R      & Is the SHA2            class of instructions                supported? \\
$\ID{AES }     $ & $           10$ & R      & Is the AES             class of instructions                supported? \\
$\ID{MASK}     $ & $            9$ & R      & Is the masking         class of instructions                supported? \\
$\ID{LEAK}     $ & $            8$ & R      & Is the leakage         class of instructions                supported? \\
$\ID{MP  }     $ & $            7$ & R      & Is the multi-precision class of instructions                supported? \\
$\ID{PACK}_{16}$ & $            6$ & R      & Is the packed          class of instructions (for $w = 16$) supported? \\
$\ID{PACK}_{8} $ & $            5$ & R      & Is the packed          class of instructions (for $w =  8$) supported? \\
$\ID{PACK}_{4} $ & $            4$ & R      & Is the packed          class of instructions (for $w =  4$) supported? \\
$\ID{PACK}_{2} $ & $            3$ & R      & Is the packed          class of instructions (for $w =  2$) supported? \\
$\ID{BIT }     $ & $            2$ & R      & Is the bit-oriented    class of instructions                supported? \\
$\ID{MEM }     $ & $            1$ & R      & Is the memory          class of instructions                supported? \\
$\ID{RND }     $ & $            0$ & R      & Is the randomness      class of instructions                supported? \\
\hline
\end{tabular}
\end{center}
\caption{A tabular      description of the $\SPR{uxcrypto}$ register.}
\label{tab:csr:uxcrypto}
\end{table}

% =============================================================================

With reference to~\cite[Chapter 2]{SCARV:RV:ISA:II},
\REFTAB{tab:csr}
outlines a number of Control and Status Registers (CSRs) which form the
(architectural) state added by \XCRYPTO to the base ISA.  The following 
\SEC[s] explain the purpose and use of each CSR.
In each diagrammatic description, note that 

\begin{enumerate}
\item Regions blocked out in black 
      (e.g.,~{\color{black}    \rule{3em}{2ex}}~)
      are deemed 
      {\em reserved};
      \XCRYPTO~{\em requires} an implementation to yield zero when the
      constituent bits of such a region are read 
      (vs. assume software will ignore them, per the RISC-V specification),
      in order to prevent any unintentionally implementation-specific 
      behaviour.

\item Regions blocked out in gray  
      (e.g.,~{\color{lightgray}\rule{3em}{2ex}}~)
      are deemed
      {\em implementation specific};
      within a general framework that governs the high-level purpose of 
      such regions, an implementation {\em may} specialise the low-level 
      semantics of constituent bits.
\end{enumerate}

% =============================================================================

\subsubsection{Class-$1$:   baseline}
\label{sec:spec:state:1}

\REFFIG{fig:csr:uxcrypto}
illustrates the 
$\SPR{uxcrypto}$
CSR; additional explanation of the constituent fields is provided in
\REFTAB{tab:csr:uxcrypto}.
$\SPR{uxcrypto}$
is a general-purpose CSR, in the sense it applies 
across the ISE as a whole 
vs.
to a specific feature class within the ISE.
The purpose of 
$\SPR{uxcrypto}$
is to support feature identification.  
In essence, 
\[
\FIELD{\SPR{uxcrypto}}{X} ~=~ \left\{\begin{array}{ll@{\;}r}
                                     0 & \mbox{feature class $X$ is} & \mbox{not supported} \\
                                     1 & \mbox{feature class $X$ is} & \mbox{    supported} \\
                                     \end{array}
                              \right.
\]
meaning that one can use field $X$ to test the presence or absence of
an associated feature class.

% =============================================================================

\subsubsection{Class-$2.3$: bit-oriented}
\label{sec:spec:state:2:3}

\REFFIG{fig:csr:ubop}
illustrates the 
$\SPR{ubop}$
CSR.
$\SPR{ubop}$
is a special-purpose CSR, in the sense it applies 
to a specific feature class within the ISE
vs.
across the ISE as a whole.
The purpose of 
$\SPR{ubop}$
is to support the
\VERB[RV]{xc.bop}
as outlined in 
\REFSEC{sec:spec:instruction:xc.bop}.
For $i \in \SET{ 0, 1 }$, the CSR encodes a function
\[
f_i : \SET{ 0, 1 }^3 \rightarrow \SET{ 0, 1 }
\]
as an $8$-entry truth-table in $\FIELD{ubop}{\ID{LUT}_{i}}$.
Said functions are used by instances of \VERB[RV]{xc.bop}, allowing the
application of a configurable $3$-input, $1$-output Boolean function.
(cf. x86 \VERB{vpternlogd}~\cite[5-446--5-468]{SCARV:X86:2:18}, and Amiga blitter~\cite[Chapter 6]{SCARV:Amiga:85})
vs. common fixed operations such as NOT, AND, OR, and XOR.

% =============================================================================

\subsubsection{Class-$2.6$: leakage}
\label{sec:spec:state:2:6}

\REFFIG{fig:csr:ufenl}
illustrates the 
$\SPR{ufenl}$
CSR.
$\SPR{ufenl}$
is a special-purpose CSR, in the sense it applies 
to a specific feature class within the ISE
vs.
across the ISE as a whole.
The purpose of 
$\SPR{ufenl}$
is to support an instance of the general FENL concept introduced by 
Gao et al.~\cite{SCARV:GMPP:20}.
Let
$
\sigma( R_i )
$
denotes the execution stage\footnote{%
A precise definition is inherently dependent on the micro-architecture, 
but it is reasonable to consider either
cycle          (for a non-pipelined case)
or
pipeline stage (for a     pipelined case)
as representative.
} some $i$-th micro-architectural resource $R_i$ (or logical group 
thereof) exists or is applied in.
A suitable choice of $R$ allows control of both 
  fine-grained 
(e.g., pipeline registers, per~\cite{SCARV:GMPP:20})
and/or
coarse-grained
(e.g., elements of the memory hierarchy, thus subsuming proposals such as that of Wistoff et al.~\cite[Section 2.4]{SCARV:WSGBH:20}).

$\SPR{ufenl}$
acts a configuration register, each $i$-th bit in which controls $R_i$: 
in essence, 
\[
\INDEX{\SPR{ufenl}}{i} ~=~ \left\{\begin{array}{ll@{\;}r}
                                  0 & \mbox{$R_i$ is} & \mbox{not flushed in execution stage $\sigma( R_i )$} \\
                                  1 & \mbox{$R_i$ is} & \mbox{    flushed in execution stage $\sigma( R_i )$} \\
                                  \end{array}
                             \right.
\]
Note that the mapping of 
$\INDEX{\SPR{ufenl}}{i}$
to 
$R_i$
is {\em inherently} implementation specific, because it depends on the
micro-architecture; this exposure of micro-architectural detail in the
ISA (or ISE in this case) is fundamental to the augmented ISA (or aISA) 
concept of
Ge et al.~\cite{SCARV:GeYarHei:18}
on which FENL is based.

% =============================================================================
