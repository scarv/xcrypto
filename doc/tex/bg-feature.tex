% =============================================================================

\subsection{Features}
\label{sec:bg:feature}

\begin{table}[t]
\begin{center}
\begin{tabular}{|l|l|cc|l|}
\hline
Description                                & Class           & Numeric ID & Mnemonic ID & Section                           \\
\hline\hline
\multirow{1}{*}{Required                 } & Baseline        & $1$        & BASE        & \REFSEC{sec:spec:instruction:1}   \\
\hline
\multirow{7}{*}{Optional, general-purpose} & Randomness      & $2.1$      & RND         & \REFSEC{sec:spec:instruction:2:1} \\
                                           & Memory          & $2.2$      & MEM         & \REFSEC{sec:spec:instruction:2:2} \\
                                           & Bit-oriented    & $2.3$      & BIT         & \REFSEC{sec:spec:instruction:2:3} \\
                                           & Packed          & $2.4$      & PACK        & \REFSEC{sec:spec:instruction:2:4} \\
                                           & Multi-precision & $2.5$      & MP          & \REFSEC{sec:spec:instruction:2:5} \\
                                           & Leakage         & $2.6$      & LEAK        & \REFSEC{sec:spec:instruction:2:6} \\
                                           & Masking         & $2.7$      & MASK        & \REFSEC{sec:spec:instruction:2:7} \\
\hline
\multirow{6}{*}{Optional, special-purpose} & AES             & $3.1$      & AES         & \REFSEC{sec:spec:instruction:3:1} \\
                                           & SHA2            & $3.2$      & SHA2        & \REFSEC{sec:spec:instruction:3:2} \\
                                           & SHA3            & $3.3$      & SHA3        & \REFSEC{sec:spec:instruction:3:3} \\
                                           & SM4             & $3.4$      & SM4         & \REFSEC{sec:spec:instruction:3:4} \\
                                           & ChaCha20        & $3.5$      & CHA         & \REFSEC{sec:spec:instruction:3:5} \\
                                           & Alzette         & $3.6$      & ALZ         & \REFSEC{sec:spec:instruction:3:6} \\
\hline
\end{tabular}
\end{center}
\caption{A list of feature classes, plus their associated numeric and mnemonic identifiers.}
\label{tab:feature}
\end{table}

Per \REFSEC{sec:bg:concept}, \XCRYPTO is a {\em meta}-extension in that it 
has 
a) one         {\em required}, baseline feature class
   which {\em must} be supported,
   plus
b) a number of {\em optional}           feature classes
   which {\em  may} be supported.
To some extent, this mirrors the extensibility afforded by RISC-V itself, 
with similar motivation: doing so allows one to tailor the ISE (resp. ISA) 
so it suits a given context, which is important because some features are 
applicable within a broad range of cryptographic workloads whereas others 
are (more) specifically applicable (even niche).

The RISC-V naming convention~\cite[Section 22]{SCARV:RV:ISA:I} for an ISE
uses a string of single-character identifiers to specify features realised
in an implementation.  We adopt a variant of this approach, wherein
\[
\XCRYPTO[$x$/$y$/$z$]
\]
denotes a concrete implementation of \XCRYPTO that supports feature classes 
$x$, $y$, and $z$: we assign a numeric and a mnemonic identifier to each
feature class (summarised by \REFTAB{tab:feature}).  For example,
\[
\XCRYPTO[1/2.1/2.2/2.3/2.4] ~\equiv~ \XCRYPTO[BASE/RND/MEM/BIT/PACK]
\]
describes an implementation which supports
a) the baseline features,
b) features relating to generation of randomness,
c) features relating to advanced memory access, namely scatter/gather
   style store and load operations,
d) features relating to bit-oriented (i.e., Boolean) operations,
   plus
e) features relating to packed (or DSP-like) arithmetic operations:
such an implementation is tailored to suit symmetric cryptography (e.g.,
AES), but {\em not} asymmetric cryptography (e.g., RSA).

% =============================================================================
