% =============================================================================

\subsection{Assumptions}
\label{sec:bg:assumption}

\begin{itemize}

\item We make {\em no} assumption about the value of $\RVXLEN$ as defined
      by the base ISA.  That is, \XCRYPTO can cater for instances where
      $
      \RVXLEN \in \SET{ 32, 64, 128 } .
      $

\item \XCRYPTO demands interaction with an RNG, concrete instantiation of 
      which is {\em unspecified}.  As such, we assume the RNG design and 
      implementation follows best-practice,
      e.g., per NIST~\cite{SCARV:NIST:SP:800_90a,SCARV:NIST:SP:800_90b,SCARV:NIST:SP:800_90c},
      and has an interface per \cite[Section 6.4]{SCARV:NIST:SP:800_90c}.

      On one hand, doing so allows flexibility in an implementation; this 
      is important, in that the RNG design and implementation will likely 
      be technology-specific 
      (e.g., differ for a given FPGA, vs. an ASIC).  
      On the other hand, however, said RNG is {\em critically} important
      wrt. security.  This implies an assumption that various challenges 
      related to the RNG design and implementation are carefully and 
      satisfactorily addressed.

\end{itemize}

% =============================================================================
