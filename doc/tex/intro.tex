% =============================================================================

Versus a more general case, cryptographic workloads are a challenge in the
sense they typically
a) require computationally intensive, somewhat niche functionality,
   and
b) form a central target in what is a complex, evolving attack surface.
The former is a particular issue, because cryptography normally represents
an enabling technology vs. a feature; put another way, it will represent
overhead when viewed from the perspective of a user.  Efficiency is hence 
a goal in and of itself, but {\em also} an enabler for security.  This is
because one cannot (or at least {\em should} not) compromise security to 
meet efficiency requirements, so delivering higher efficiency can also be 
pitched as an enabler for countermeasures against attack (since there will
be a greater margin within which to do so).

This document acts as the specification for a 
non-standard extension~\cite[Section 21.1]{SCARV:RV:ISA:I} 
to any one of the RISC-V base ISAs
(e.g., RV32I~\cite[Section 2]{SCARV:RV:ISA:I}, RV32E~\cite[Section 3]{SCARV:RV:ISA:I}, RV64I~\cite[Section 4]{SCARV:RV:ISA:I}, or RV128I~\cite[Section 5]{SCARV:RV:ISA:I}),
which we dub \XCRYPTO; it forms an output from the SCARV\footnote{
\url{http://www.scarv.org}, \url{http://www.github.com/scarv}
} project, funded by EPSRC\footnote{
\url{http://gow.epsrc.ac.uk/NGBOViewGrant.aspx?GrantRef=EP/R012288/1}
} under the RISE\footnote{
\url{http://www.ukrise.org}
} programme.  
\XCRYPTO aims to enable
a) efficient
   and
b) secure
software implementation of cryptographic primitives, within a remit which
is conceptually analogous to that of the standard F and D
(i.e., floating-point) 
extensions.
Note that although the document {\em is} a specification for \XCRYPTO, it
categorically {\em is not} an implementation guide.  In order to avoid the 
document body becoming too verbose, we use 
\REFAPPX{appx:notation}
and
\REFAPPX{appx:related}
to capture a detailed description of the notation used and related work
respectively; associated content such as
a) a set of design notes
   and
b) a changelog\footnote{
   \url{http://keepachangelog.com}
   } 
is maintained within the repository wiki vs. in the document.

% -----------------------------------------------------------------------------

The \XCRYPTO project as a whole follows a semantic 
versioning\footnote{
\url{http://semver.org}
} (or major/minor/patch) convention.
Over time, \XCRYPTO has evolved along various branches 
(each identified by an associated major version):

\begin{itemize}
\item The (now abandoned) 
      $0.x.y$ branch
      represents an initial prototype.
      It can be characterised as deliberately disjoint from the RISC-V 
      base ISA(s), and so, in concept, aligned with implementation as a 
      separate co-processor.
\item The (active)
      $1.x.y$ branch
      represents a subsequent refinement.
      It can be characterised as taking the functionality from $0.x.y$, 
      but integrating inline with vs. alongside the RISC-V base ISA(s), 
      i.e., in the form of a conventional ISA vs. a co-processor.
\end{itemize}

\noindent
The long-term goal is to develop the specification, using it as a means 
for experimentation, and, ultimately, the basis for a standard (i.e., 
``official'') RISC-V extension proposal.
That said, statements such as 
``\XCRYPTO is   X'' 
or
``\XCRYPTO does Y''
should be carefully qualified with {\em currently} vs. {\em definitively}.  
In particular, we expect some degree of iteration and so change to emerge 
from work in progress wrt. design, implementation, and evaluation.

% =============================================================================
