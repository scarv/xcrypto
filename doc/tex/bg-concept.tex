% =============================================================================

\subsection{Concept}
\label{sec:bg:concept}

\begin{itemize}

\item Assuming carte blanche wrt. design and implementation of some processor
      core, and the wider system it is coupled to, a significant design space 
      of candidate approaches exists.  For example, one could realise a given 
      cryptographic primitive using
      a) hardware-only,
      b) mixed (or hybrid),
         or
      c) software-only
      techniques.

      The remit of \XCRYPTO is strictly focused on providing support for the
      latter techniques; one could view said remit as focused on maximising 
      the viability of software-based cryptographic implementations.  It has
      an implicit focus on constrained, e.g., micro-controller class, cores, 
      although the concept and utility is arguably broader.
      Various published work is emerging that motivates and/or evidences such
      an approach: in the specific context of RISC-V,
      see, e.g.,~\cite{SCARV:Stoffelen:19}.

\item The following design criteria rationalise this approach, some of which
      are related (or even stem directly from one and other):

      \begin{itemize}
      \item {\bf         Security}.
            An unfortunate fact is that security will commonly be relegated 
            to a second-class design metric, and so, by implication, viewed 
            as being of secondary importance;
            see, e.g., \cite{SCARV:Lee:03,SCARV:RKLMR:03,SCARV:RRKH:04,SCARV:BurMutTiw:16}.
            Modulo the limitations as an ISE, \XCRYPTO contrasts by treating
            security as a first-class metric and so at least as important as 
            more traditional alternatives.
            For example,
            within the context of \XCRYPTO we deem it reasonable to trade-off 
            improved security vs. degradation of instruction throughput.
      \item {\bf      Consistency}.
            As far as reasonable, \XCRYPTO is consistent with the overarching
            RISC-V philosophy, and associated base ISA.  Doing so will demand 
            considered compromises vs. a clean-slate design, but, equally,
            should maximise the resulting utility.
            For example,
            we attempt to minimise 
            a) deviation from the existing instruction encoding formats,
               and 
            b) introduction of additional state.
      \item {\bf       Generality}.
            \XCRYPTO aims to be 
            general-purpose,
            in the sense it attempts to avoid
            inclusion of overly functionality-specific features.
      \item {\bf      Flexibility}. 
            \XCRYPTO aims to be
            flexible,
            in the sense it attempts to avoid
            ``baked in'' (or hard coded) trade-offs.
            Put another way, although the design {\em will} imply trade-offs 
            (e.g., vs. efficiency, typically delivered by hardware-only techniques), 
            it {\em should} more easily support
            a) agility wrt. primitive, algorithm, and parameter choices,
               and
            b) instrumentation of context-dependent countermeasures.
      \item {\bf    Composability}.
            To mitigate a given attack, a layered approach (cf. defence in 
            depth) is normally preferred: this favours the use of multiple
            countermeasures, vs. a single, perfect panacea.  The same form
            of argument applies to efficiency, in the sense that efficiency
            requirements may render software-only implementation techniques,
            even {\em with} support of \XCRYPTO, insufficient.
            As such, \XCRYPTO
            a) should be viewed as one option for or layer in a solution,
               and
            b) prefers features that can co-exist over those which cannot.
      \item {\bf Implementability}. 
            It should be possible to implement \XCRYPTO with as little
            a) overhead   (e.g., wrt. additional logic),
               and
            b) difficulty (e.g., wrt. outright complexity, or complicating factors such as verification)
            as possible.  This implies a preference for features that avoid 
            a {\em requirement} for complex hardware or invasive alteration 
            to the host core.
      \item {\bf    Measurability}.
            Given the remit of extending a base ISA, any feature in \XCRYPTO 
            should offer reproducible, demonstrable value vs. this baseline.
            For a given feature, this goal should be supported by provision 
            of associated reference implementations of cryptographically 
            interesting benchmark kernels.
      \end{itemize}

\item Aligning with several cited design criteria, \XCRYPTO can be viewed as
      a form of {\em meta}-extension: it is organised into a set of feature 
      {\em classes} (or sub-extensions), which can be selected from to suit.
      Note that
      \REFSEC{sec:bg:feature} 
      provides a high-level description of and motivation for said feature
      classes, and an associated CSR described in 
      \REFSEC{sec:spec:state}
      facilitates feature identification.

\item All that said, \XCRYPTO may naturally be deemed {\em in}appropriate 
      for some use-cases; acknowledging this fact, we carefully pitch it 
      as {\em an} approach rather than {\em the} approach.

\end{itemize}

% =============================================================================
