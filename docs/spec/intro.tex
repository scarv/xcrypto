% =============================================================================

Versus a general case, cryptographic workloads are challenging in that they
typically
a) require computationally intensive, somewhat niche functionality,
   and
b) form a central target in what is a complex, evolving attack surface.
The former is a particular issue, in the sense cryptography is typically an
enabling technology vs. a feature per se: from the perspective of the user,
it represents pure overhead.  Efficiency is a goal in and of itself, but is
also as an enabler for security: one cannot (or at least {\em should} not)
compromise security to meet efficiency requirements, so delivering higher
efficiency can be pitched as an enabler for countermeasures against attack
(since there is more margin in which to do so).

This document acts as a specification for a 
greenfield, non-standard extension~\cite[Section 21.1]{SCARV:RV:ISA:I:17} 
to the RISC-V 
RV32I~\cite[Section 2]{SCARV:RV:ISA:I:17}
or
RV32E~\cite[Section 3]{SCARV:RV:ISA:I:17}
base 
ISA, which we dub \ISE.
The \ISE ISE aims to support software implementations of both symmetric and 
asymmetric cryptographic primitives.  By analogy, one could view this remit
as similar to that of a floating-point co-processor, but for cryptography: 
per the above, it aims to enable
a) efficient
   and
b) secure
execution of said implementations.  At a high level, (pertinent) features 
that enable it does so can be classified as follows:

\begin{itemize}
\item Class-$1$: randomness.
\item Class-$2$: packed                  operations.
\item Class-$3$: bit-wise                operations.
\item Class-$4$: multi-precision integer operations.
\end{itemize}

\noindent
Note that this document {\em is} a specification for \ISE, but {\em is not} 
an implementation guide: we provide a separate document for that purpose.
Also note that it should be viewed as a prototype or draft, so statements 
such as 
``\ISE is   X'' 
or
``\ISE does Y''
should be carefully qualified with {\em currently} vs. {\em definitively}.  
In particular, we expect some degree of fine-tuning to emerge from ongoing
implementation and evaluation effort.

% =============================================================================
