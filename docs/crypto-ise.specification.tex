
This section specifies the ISE in terms of it's state and the instructions
which operate on that state.

\subsection{Notation}

Lorem ipsum doler set amet.


\subsection{State}

The ISE introduces an additional $16$ by $32$-bit register file which
is used exclusively by the instructions in this ISE. It is to the
crypto ISE what the floating-point register file is to the RISC-V
F extension.

The addressing is shown diagrammatically in table \ref{tab:state-addr}.

\begin{table}[h!]
\centering
\begin{tabular}{|l|l l l l|}
\hline
\multicolumn{1}{|l|}{3:0} & \textbf{ } & \textbf{ } & \textbf{ } & \textbf{ } \\ \hline
\textbf{0 }   & r0.3       & r0.2       & r0.1       & r0.0       \\ \hline
\textbf{1 }   & r1.3       & r1.2       & r1.1       & r1.0       \\ \hline
\textbf{2 }   & r2.3       & r2.2       & r2.1       & r2.0       \\ \hline
\textbf{3 }   & r3.3       & r3.2       & r3.1       & r3.0       \\ \hline
\textbf{4 }   & r4.3       & r4.2       & r4.1       & r4.0       \\ \hline
\textbf{5 }   & r5.3       & r5.2       & r5.1       & r5.0       \\ \hline
\textbf{6 }   & r6.3       & r6.2       & r6.1       & r6.0       \\ \hline
\textbf{7 }   & r7.3       & r7.2       & r7.1       & r7.0       \\ \hline
\textbf{8 }   & r8.3       & r8.2       & r8.1       & r8.0       \\ \hline
\textbf{9 }   & r9.3       & r9.2       & r9.1       & r9.0       \\ \hline
\textbf{10}   & r10.3      & r10.2      & r10.1      & r10.0      \\ \hline
\textbf{11}   & r11.3      & r11.2      & r11.1      & r11.0      \\ \hline
\textbf{12}   & r12.3      & r12.2      & r12.1      & r12.0      \\ \hline
\textbf{13}   & r13.3      & r13.2      & r13.1      & r13.0      \\ \hline
\textbf{14}   & r14.3      & r14.2      & r14.1      & r14.0      \\ \hline
\textbf{15}   & r15.3      & r15.2      & r15.1      & r15.0      \\ \hline
\end{tabular}
\caption{This table shows the per-byte layout of the register file. Note
that there is no {\em zero} register in the ISE. The register with
address $0$ will preserve writes.}
\label{tab:state-addr}
\end{table}

Unlike the RISC-V $\GPR$s, the zeroth $\CPR$ is not tied to zero.

\note{
A smart programmer will remember to clear their $\CPR$s of any secret data
before returning execution to some part of the system they did not write.
}

\subsubsection{ABI Standards}

All of the state added in the Crypto ISE is considered {\em callee save}
for the purposes of the ABI. That is, if {\tt function1} calls 
{\tt function2}, then {\tt function2} is responsible for saving to the
stack any $\CPR$ registers it needs, and popping them off before returning.

The $\CPR$ registers $0..4$ are considered as function arguments
and/or return values. 
All other $\CPR$ registers are considered temporaries.

\subsection{Instructions}

All instructions occupy the \encspace encoding space of the RISC-V opcode
map, using the associated \encopcode prefix for bits $6:0$ of each 
instruction.

\subsubsection{Move Instructions}
\paragraph{MV2GPR}

Move a $\CPR$ register to a $\GPR$ register.

\begin{itemize}
\item \INST{MV2GPR gpr, cpr}{
    $\GPR[*][\OPER{gpr}] \ASN \CPR[*][\OPER{cpr}]$
}
\end{itemize}


\paragraph{MV2COP}

The complement of \ASM{MV2GPR}.
Move a $\GPR$ register to a $\CPR$ register.

\begin{itemize}
\item \INST{MV2COP cpr, gpr}{
    $\CPR[*][\OPER{cpr}] \ASN \GPR[*][\OPER{gpr}]$
}
\end{itemize}

\paragraph{LUI.cr}

The \ASM{LUI.cr} instruction loads a 16 bit immediate into the
upper half of a $\CPR$ register. The other half of the
destination register is set to zero.

\begin{itemize}
\item \INST{LUI.cr rd, imm16 }{
    $\CPR[*][\OPER{rd}][31:16] \ASN imm16$ \\
    $\CPR[*][\OPER{rd}][15: 0] \ASN 0x0$
}
\end{itemize}

\paragraph{ORI.cr}

The \ASM{ORI.cr} instruction bitwise or's a 16 bit immediate into the
lower half of a $\CPR$ register. 

\begin{itemize}
\item \INST{ORI.cr rd, imm16 }{
    $\CPR[*][\OPER{rd}] \ASN \CPR[*][\OPER{rd}] \IOR imm16$ \\
}
\end{itemize}

\subsubsection{Packed Arithmetic}

The packed arithmetic instructions are designed to provide scalable, sub-word
packed operatons on multiple independent values. They operate on words, and
chunk them into 1, 2, 4, 8, 16 or 32 separate values as shown in the table
below.

These instructions are designed to allow fast and efficient implementations
of the highly data parallel operations found in cryptographic block ciphers.

The general operation of each instruction is shown below, with $f(x,y)$
used to represent the specific functionality of each instruction.

\medskip

\begin{algorithm}[H]
    \KwIn{rs1, rs2 - Source registers 1 and 2}
    \KwIn{rd - destination register}
    \KwIn{pw - Pack width, the bit width of each operation}
    $b   \ASN 0 $\;
\While {$b < {\bf W}$}{
$\CPR[*][rd][b+pw:b] \ASN f(\CPR[*][rs1][b+pw:b],\CPR[*][rs2][b+pw:b])$\;
$b \ASN b + pw$\;
    }
\caption{Algorithm for performing packed arithmetic on words in the 
$\CPR$ register file. Here, ${\bf W}$ is the word-width in bits of the
registers, assumed to be $32$. For example, if $pw$ is 32, then the loop
will execute once, applying the function to the full 32-bit word. If $pw$
were $8$, then the function would be applied $8$, once to each corresponding
pair of nibbles in the word.}
\end{algorithm}

\paragraph{ADD.px}

The instruction \ASM{ADD.px} treats the source and destination registers as
1,2,4,8,16 or 32 unsigned packed fields, and adds corresponding fields
together ignoring any carry bits.

No overflow conditions are tested for, and \ASM{ADD.px} raises no exceptions.

\paragraph{SUB.px}

The instruction \ASM{SUB.px} treats the source and destination registers as
1,2,4,8,16 or 32 unsigned, packed fields, and subtracts the {\tt rs2} fields
from the corresponding {\tt rs1} fields.

No overflow conditions are tested for, and \ASM{SUB.px} raises no exceptions.

\paragraph{MUL.px}

The instruction \ASM{MUL.px} treats the source and destination registers as
1,2,4,8,16 or 32 unsigned, packed fields, and multiplies the {\tt rs2} fields
by the corresponding {\tt rs1} fields.

No overflow conditions are tested for, and \ASM{MUL.px} raises no exceptions.

\paragraph{SLL.px}

The instruction \ASM{SLL.px} treats the source and destination registers as
1,2,4,8,16 or 32 unsigned, packed fields, and shifts the {\tt rs2} fields
left by the value in {\tt SHAMT}.

If the distance to shift by is greater than or equal to the size of each
packed field, then the result of the \ASM{SLL.px} instruction is
{\em implementation dependent}.

\paragraph{SRL.px}

The instruction \ASM{SRL.px} treats the source and destination registers as
1,2,4,8,16 or 32 unsigned, packed fields, and shifts the {\tt rs2} fields
right by the value in {\tt SHAMT}.

If the distance to shift by is greater than or equal to the size of each
packed field, then the result of the \ASM{SRL.px} instruction is
{\em implementation dependent}.

\paragraph{ROT.px}

The instruction \ASM{ROT.px} treats the source and destination registers as
1,2,4,8,16 or 32 unsigned, packed fields, and rotates the {\tt rs2} fields
left by the value in {\tt SHAMT}.

\subsubsection{Bitwise and Twiddle Instructions}

\paragraph{XOR.cr, AND.cr, OR.cr}

The \ASM{XOR.cr}, \ASM{AND.cr}, and \ASM{OR.cr} instructions compute the
appropriate bitwise result of their operation between the {\tt rs1} and
{\tt rs2} operands. The result is then written into {\tt rd}.
\designnote{Originally the bitwise instructions were to have twiddle
functionality built-in, removing the need for a dedicated twiddle instruction.
The encoding space required however did not fit well with the existing RISC-V
register address and function field layout, and was extremely expensive.}

The only difference between these instructions and their standard RISC-V
counterparts is that they operate on the $\CPR$ register file.

\begin{itemize}
\item \INST{XOR.cr rd, rs1, rs2}{
    $ \CPR[*][\OPER{rd}] \ASN \CPR[*][rs1] \XOR \CPR[*][rs2] $ \\
}
\item \INST{AND.cr rd, rs1, rs2}{
    $ \CPR[*][\OPER{rd}] \ASN \CPR[*][rs1] \AND \CPR[*][rs2] $ \\
}
\item \INST{OR.cr rd, rs1, rs2}{
    $ \CPR[*][\OPER{rd}] \ASN \CPR[*][rs1] \IOR \CPR[*][rs2] $ \\
}
\end{itemize}

Note that by having {\tt rs1} the same as {\tt rs2}, the twiddle
arithmetic instructions can be used to arbitrarily re-arrange bytes
within words. They can also be used to duplicate bytes within words.

\paragraph{TWID.B, TWID.HN, TWID.LN}

The \ASM{TWID.B} instruction is used to {\em twiddle} bytes within words.
It arbitrarily maps the bytes of a source word onto the bytes of a
destination word.

The \ASM{TWID.[HN,LN]} instruction is used to {\em twiddle} nibbles within
the high and low half-words of a word respectively.

\begin{itemize}
\item \INST{TWID.B rd, rs1, b0, b1, b2, b3}{
    $ \CPR[*][\OPER{rd}][0] \ASN \CPR[*][\OPER{rs1}][\OPER{b0}] $ \\
    $ \CPR[*][\OPER{rd}][1] \ASN \CPR[*][\OPER{rs1}][\OPER{b1}] $ \\
    $ \CPR[*][\OPER{rd}][2] \ASN \CPR[*][\OPER{rs1}][\OPER{b2}] $ \\
    $ \CPR[*][\OPER{rd}][3] \ASN \CPR[*][\OPER{rs1}][\OPER{b3}] $ 
}
\item \INST{TWID.LN rd, rs1, n0, n1, n2, n3}{
    $ nibbles \ASN \TUPLE{\CPR[*][\OPER{rd}] [15:12],
                          \CPR[*][\OPER{rd}] [11: 8],
                          \CPR[*][\OPER{rd}] [ 7: 4],
                          \CPR[*][\OPER{rd}] [ 3: 0]} $ \\
    $ \CPR[*][\OPER{rd}] [ 3: 0] \ASN nibbles[\OPER{n0}] $ \\
    $ \CPR[*][\OPER{rd}] [ 7: 4] \ASN nibbles[\OPER{n1}] $ \\
    $ \CPR[*][\OPER{rd}] [11: 8] \ASN nibbles[\OPER{n2}] $ \\
    $ \CPR[*][\OPER{rd}] [15:12] \ASN nibbles[\OPER{n3}] $ \\
    $ \CPR[*][\OPER{rd}] [31:16] \ASN \CPR[*][\OPER{rs1}] [31:16] $
}
\item \INST{TWID.HN rd, rs1, n0, n1, n2, n3}{
    $ nibbles \ASN \TUPLE{\CPR[*][\OPER{rd}] [31:28],
                          \CPR[*][\OPER{rd}] [27:24],
                          \CPR[*][\OPER{rd}] [23:20],
                          \CPR[*][\OPER{rd}] [19:16]} $ \\
    $ \CPR[*][\OPER{rd}] [19:16] \ASN nibbles[\OPER{n0}] $ \\
    $ \CPR[*][\OPER{rd}] [23:20] \ASN nibbles[\OPER{n1}] $ \\
    $ \CPR[*][\OPER{rd}] [27:24] \ASN nibbles[\OPER{n2}] $ \\
    $ \CPR[*][\OPER{rd}] [31:28] \ASN nibbles[\OPER{n3}] $ \\
    $ \CPR[*][\OPER{rd}] [15: 0] \ASN \CPR[*][\OPER{rs1}] [15: 0] $
}
\end{itemize}

Twiddle can be used to broadcast and re-order bytes within words.


\subsubsection{Multi-precision Arithmetic}

Public key cryptography relies heavily on modular arithmetic over very
large integers. The following instructions are designed to allow
operations over arbitrarily large integers, using software based carry
and overflow mechanisms. This mirrors how the base RISC-V ISA currently
supports carry and overflow.

These instructions use two destination registers and between three
and four source registers. Hence, some restrictions are placed on how
registers can be addressed.

\begin{itemize}
\item Destination registers are referred too as contiguous, even-odd
pairs. This means any two adjacently addressed registers, where the
first register address is even (zero being considered even for this
discussion) and the second address is one more than the first can be
used as the destination for a multi-precision instructions. \designnote{
    This scheme means the function to compute the actual source registers
    from the minimum 3-bit encoding is simply to append either {\tt 00}
    or {\tt 01} depending on the even or odd address respectively.
}

For example, $(0,1), (2,3)$ and $(10,11)$ are all valid
pairs of destination registers. $(5,6)$ is not valid because the first
is an odd number. $(6,5)$ is not valid because $6 > 5$. The pair $(6,9)$
is not valid because $6$ and $9$ are not adjacent.
\end{itemize}

\paragraph{ADD.mp}

The \ASM{ADD.mp} instruction takes three source registers and two
destination registers. The intermediate result $t_0$ is a $32$
bit integer, comprising the unsigned sum of the $32$ bit values in
{\tt rs1} and {\tt rs2}. The final $33$ bit result $t_1$ is the signed
sum of $t_0$ and the $32$ bit value in $rs3$.

The low $32$ bits of $t_1$ are written to {\tt rd1}, while the carry out
bit is written to {\tt rd2}.

\begin{itemize}
\item \INST{ADD.mp (rd1, rd2), rs1, rs2, rs3     }{
$t \ASN ( \CPR[*][\OPER{rs}_1] \SOP{+}{u}     \CPR[*][\OPER{rs}_2] ) + \CPR[*][\OPER{rs}_3] $\\
$\CPR[*][\OPER{rd}_1] \ASN t_{1 \cdot \W - 1, \ldots,  0}$\\
$\CPR[*][\OPER{rd}_2] \ASN t_{        \W                }$\\
}
\end{itemize}

\paragraph{SUB.mp}

\begin{itemize}
\item \INST{SUB.mp (rd1, rd2), rs1, rs2, rs3     }{
$t                  \ASN ( \CPR[*][\OPER{rs}_1] \SOP{+}{u}     \CPR[*][\OPER{rs}_2] ) - \CPR[*][\OPER{rs}_3]$\\
$\CPR[*][\OPER{rd}_1] \ASN t_{1 \cdot \W - 1, \ldots,  0}$\\
$\CPR[*][\OPER{rd}_2] \ASN t_{        \W                }$\\
}
\end{itemize}

\paragraph{ACC.mp}

\begin{itemize}
\item \INST{ACC.mp (rd1, rd2), rs1, rs2, rs3     }{
$t \ASN ( \CPR[*][\OPER{rs}_1] \CONS \CPR[*][\OPER{rs}_0] ) + \CPR[*][\OPER{rs}_2] + \CPR[*][\OPER{rs}_3] $\\
$\CPR[*][\OPER{rd}_1] \ASN t_{1 \cdot \W - 1, \ldots,  0} $\\
$\CPR[*][\OPER{rd}_2] \ASN t_{2 \cdot \W - 1, \ldots, \W} $\\
}
\end{itemize}

\paragraph{MAC.mp}

\begin{itemize}
\item \INST{MAC.mp (rd1, rd2), rs1, rs2, rs3}{
$t                  \ASN ( \CPR[*][\OPER{rs}_1] \SOP{\cdot}{u} \CPR[*][\OPER{rs}_2] ) + \CPR[*][\OPER{rs}_3]$\\
$\CPR[*][\OPER{rd}_0] \ASN t_{1 \cdot \W - 1, \ldots,  0}$\\
$\CPR[*][\OPER{rd}_1] \ASN t_{2 \cdot \W - 1, \ldots, \W}$\\
}
\end{itemize}

\paragraph{SLL.mp}

\begin{itemize}
\item \INST{SLL.mp (rd1, rd2), rs1, rs2, rs3     }{
$\CPR[*][\OPER{rd}_2] \ASN \CPR[*][\OPER{rs}_1] \LSH      \CPR[*][\OPER{rs}_2]_{\log_2(\W)-1 \ldots 0} \IOR \CPR[*][\OPER{rs}_3]$\\
$\CPR[*][\OPER{rd}_1] \ASN \CPR[*][\OPER{rs}_1] \RSH \W - \CPR[*][\OPER{rs}_1]_{\log_2(\W)-1 \ldots 0}$\\
}
\end{itemize}

\paragraph{SRL.mp}

\begin{itemize}
\item \INST{SRL.mp (rd1, rd2), rs1, rs2, rs3     }{
$\CPR[*][\OPER{rd}_2] \ASN \CPR[*][\OPER{rs}_1] \LSH \W - \CPR[*][\OPER{rs}_2]_{\log_2(\W)-1 \ldots 0}$\\
$\CPR[*][\OPER{rd}_1] \ASN \CPR[*][\OPER{rs}_1] \RSH      \CPR[*][\OPER{rs}_2]_{\log_2(\W)-1 \ldots 0} \IOR \CPR[*][\OPER{rs}_3]$\\
}
\end{itemize}

\paragraph{EQU.mp}

\begin{itemize}
\item \INST{EQU.mp rd1,      rs1, rs2, rs3     }{
$\GPR[*][\OPER{rd}_1] \ASN ( \CPR[*][\OPER{rs}_1] = \CPR[*][\OPER{rs}_2] ) \AND \CPR[*][\OPER{rs}_3]  $\\
}
\end{itemize}

\paragraph{LTU.mp}

\begin{itemize}
\item \INST{LTU.mp rd1,      rs1, rs2, rs3     }{
$\GPR[*][\OPER{rd}_1] \ASN ( \CPR[*][\OPER{rs}_1] < \CPR[*][\OPER{rs}_2] ) \IOR ( ( \CPR[*][\OPER{rs}_1] = \CPR[*][\OPER{rs}_2] ) \AND \CPR[*][\OPER{rs}_3] )$
}
\end{itemize}

\paragraph{GTU.mp}

\begin{itemize}
\item \INST{GTU.mp rd1,      rs1, rs2, rs3     }{
$\GPR[*][\OPER{rd}_1] \ASN ( \CPR[*][\OPER{rs}_1] > \CPR[*][\OPER{rs}_2] ) \IOR ( ( \CPR[*][\OPER{rs}_1] = \CPR[*][\OPER{rs}_2] ) \AND \CPR[*][\OPER{rs}_3] )$
}
\end{itemize}

\subsubsection{Memory Access}

\paragraph{LBU.cr, LHU.cr, LW.cr}

The \ASM{LBU.cr}, \ASM{LHU.cr} and \ASM{LW.cr} instructions are analogous to
the standard RISC-V variants but with the following differences:

\begin{itemize}
\item The halfword and byte variants can specify which halfword or byte of
      the destination register they are targeting.
\item There is no sign-extending equivilent of the \ASM{LH} and \ASM{LB}
      instructions.
\item The base address is taken from the normal RISC-V $\GPR$ set, while the
      target register is always in the $\CPR$ set.
\end{itemize}

These instructions will raise a {\em load address misaligned} instruction
if: the final source address is not aligned to the data type the instruction
is loading {\em and} the implementation does not support misaligned accesses.
The general rule here is that the \ASM{L*.cr} instructions should behave in
the same way as their RISC-V cousins in terms of exception handling,
interrupt behaviour and atomicity rules.

\begin{itemize}
\item \INST{LBU.cr  rd, b, imm(rs1)}{
    $addr \ASN  \GPR[*][rs1] + signextend(imm, 32)$ \\
    $\CPR[*][rd][b] \ASN \MEM[*][addr]$
}
\item \INST{LHU.cr  rd, h, imm(rs1)}{
    $addr \ASN  \GPR[*][rs1] + signextend(imm, 32)$ \\
    $b_1 \ASN  (2'00_2 \IOR h) \LSH 1$ \\
    $b_2 \ASN  b_1 \IOR 1'1_2$ \\
    $\CPR[*][rd][b_2] \ASN \MEM[*][addr + 1]$ \\
    $\CPR[*][rd][b_1] \ASN \MEM[*][addr + 0]$
}
\item \INST{LW.cr  rd, imm(rs1)}{
    $addr \ASN  \GPR[*][rs1] + signextend(imm, 32)$ \\
    $\CPR[*][rd][rb + 3] \ASN \MEM[*][addr+3]$ \\
    $\CPR[*][rd][rb + 2] \ASN \MEM[*][addr+2]$ \\
    $\CPR[*][rd][rb + 1] \ASN \MEM[*][addr+1]$ \\
    $\CPR[*][rd][rb + 0] \ASN \MEM[*][addr+0]$
}
\end{itemize}

\note{
    The notation above assumes a byte addressed memory, where {\tt db}
    is a two bit byte address and {\tt dh} is a one bit halfword address.
}

\paragraph{SB.cr, SH.cr, SW.cr}

The \ASM{SB.cr}, \ASM{SH.cr} and \ASM{SW.cr} instructions are analogous to
the standard RISC-V variants but with the following differences:

\begin{itemize}
\item The halfword and byte variants can specify which halfword or byte of
      the source register they store.
\item The base address is taken from the normal RISC-V $\GPR$ set, while the
      source data / destination register is always in the $\CPR$ set.
      \designnote{
        The decision to source the base address from the $\GPR$ set rather
        than a $\CPR$ was because all of the RISC-V address calculation
        instructions work perfectly well on the $\GPR$s and there is no sense
        in duplicating that functionality for this ISE. This free's up
        encoding space, and lets the $\CPR$ state be used for the compute
        work it was designed for rather than mundane address calculation.
        While care must be
        taken to avoid hazards between the difference register sets, we
        belive this decision makes sense overall.
      }
\end{itemize}

These instructions will raise a {\em store address misaligned} exception
if: the final source address is not aligned to the data type the instruction
is storing {\em and} the implementation does not support misaligned accesses.
The general rule here is that the \ASM{L*.cr} instructions should behave in
the same way as their RISC-V cousins in terms of exception handling,
interrupt behaviour and atomicity rules.

\begin{itemize}
\item \INST{SB.cr  rs1, b, imm(rs2)}{
    $addr \ASN  \GPR[*][rs1] + signextend(imm, 32)$ \\
     $\MEM[*][addr] \ASN \CPR[*][rs2][b]$
}
\item \INST{SH.cr  rs1, h, imm(rs2)}{
    $addr \ASN  \GPR[*][rs1] + signextend(imm, 32)$ \\
    $b_1 \ASN  (2'00_2 \IOR h) \LSH 1$ \\
    $b_2 \ASN  b_1 \IOR 1'1_2$ \\
    $\MEM[*][addr + 1] \ASN \CPR[*][rs2][b_2]$ \\
    $\MEM[*][addr + 0] \ASN \CPR[*][rs2][b_1]$
}
\item \INST{SW.cr  rs1, imm(rs2)}{
    $addr \ASN  \GPR[*][rs1] + signextend(imm, 32)$ \\
    $\MEM[*][addr+3] \ASN \CPR[*][rd][rb + 3]$ \\
    $\MEM[*][addr+2] \ASN \CPR[*][rd][rb + 2]$ \\
    $\MEM[*][addr+1] \ASN \CPR[*][rd][rb + 1]$ \\
    $\MEM[*][addr+0] \ASN \CPR[*][rd][rb + 0]$
}
\end{itemize}

\note{
    The notation above assumes a byte addressed memory, where {\tt b}
    is a two bit byte address and {\tt h} is a one bit halfword address.
}

\paragraph{SCATTER, GATHER}

The \ASM{SCATTER} and \ASM{GATHER} instructions are designed to be used as
s-box lookup instructions. They allow four separate lookups per instruction
into a 256 entry LUT stored in memory. \ASM{GATHER} will move the looked-up
values from memory into a register, while \ASM{SCATTER} will move bytes of a
register word back into the LUT in memory.

The address $a_x$ of each byte to be loaded or stored is computed by
adding an offset byte $x$ from the {\tt offsets} $\CPR$ register, and
adding it to the $base$ $\GPR$ register.

Both instructions use the $\GPR$s to source their base address, while the
offset indices into the LUT come from the $\CPR$ registers. This mirrors
how the ISE load and store instructions compute their addresses.

\begin{itemize}
\item \INST{GATHER rd, base, offsets}{
    $ a_0 \ASN \GPR[*][base] + \CPR[*][offsets][0]$ \\
    $ a_1 \ASN \GPR[*][base] + \CPR[*][offsets][1]$ \\
    $ a_2 \ASN \GPR[*][base] + \CPR[*][offsets][2]$ \\
    $ a_3 \ASN \GPR[*][base] + \CPR[*][offsets][3]$ \\
    $\CPR[*][rd][0] \ASN \MEM[*][a_0]$ \\
    $\CPR[*][rd][1] \ASN \MEM[*][a_1]$ \\
    $\CPR[*][rd][2] \ASN \MEM[*][a_2]$ \\
    $\CPR[*][rd][3] \ASN \MEM[*][a_3]$
}
\item \INST{SCATTER rs, base, offsets}{
    $ a_0 \ASN \GPR[*][base] + \CPR[*][offsets][0]$ \\
    $ a_1 \ASN \GPR[*][base] + \CPR[*][offsets][1]$ \\
    $ a_2 \ASN \GPR[*][base] + \CPR[*][offsets][2]$ \\
    $ a_3 \ASN \GPR[*][base] + \CPR[*][offsets][3]$ \\
    $ \MEM[*][a_0] \ASN \CPR[*][rs][0]$ \\
    $ \MEM[*][a_1] \ASN \CPR[*][rs][1]$ \\
    $ \MEM[*][a_2] \ASN \CPR[*][rs][2]$ \\
    $ \MEM[*][a_3] \ASN \CPR[*][rs][3]$
}
\end{itemize}

\note{
    The scatter instruction requires three register file reads.
    Because the reads are split across the $\GPR$ and $\CPR$ registers,
    no {\em extra} ports are needed on either register file beyond the
    standard two-read-one-write set.
}

An implementation may perform the memory accesses in any order. The accesses
appear as {\em byte} accesses, so the value of the {\em base} register
need not be word or halfword aligned.

If any of the memory accesses cause
a {\em load access fault, load page fault, store page fault} or
{\em store access fault} then the address of the access which caused the
exception is written to the {\tt MTVAL} csr. The value taken by the
destination register bytes should an exception occur at any point in the
instruction is {\em implementation dependent}.

Implementations may abort execution of \ASM{SCATTER} or \ASM{GATHER} part way
though in order to service an interrupt. If the instructions are aborted
early, the value taken by the destination register bytes is {\em
implementation dependent}. The implementation is responsible for ensuring
that the {\tt EPC} register is set appropriately depending on whether the
interrupt is taken during, or at the end of, the instruction.
\designnote{Implementation of these instructions consider the impact of
shared memory systems, since it will be possible for other agents in the
system to access memory during the execution of this instruction. In systems
possessing a cache, programmers must be mindful that these instructions
can cause odd behaviour due to the non-linear access pattern the instruction
creates. All accesses will be within 256 bytes of one another however,
which may motivate particular alignment of the LUT based on cache-line size}

The rationale for these instructions is two fold. They help with code density,
since at-least four instructions worth of work (eight in RV32) can now be done
in one instruction. They can also help with energy efficiency, since fewer
instructions need to be fetched and travel through an execution pipeline in
order to achieve the same goal. They are not designed to enhance performance
(in terms of cycles to complete an operation), but this may be a side-effect
of some implementations.

\newpage
\subsection{Instruction Listings}

This section lists the assembly notation for the instructions, as well as
their encodings.

{\tt
MV2GPR  rd, rs1                 \\
MV2COP  rd, rs1                 \\
TWID.B  rd, rs1, b0, b1, b2, b3 \\
TWID.HN rd, rs1, n0, n1, n2, n3 \\
TWID.LN rd, rs1, n0, n1, n2, n3 \\
ADD.px  rd, rs1, rs2            \\
SUB.px  rd, rs1, rs2            \\
MUL.px  rd, rs1, rs2            \\
SLL.px  rd, rs1, shamt          \\
SRL.px  rd, rs1, shamt          \\
ROT.px  rd, rs1, shamt          \\
XOR.cr  rd, rs1, rs2            \\
AND.cr  rd, rs1, rs2            \\
OR.cr   rd, rs1, rs2            \\
ADD.mp  (rd1, rd2),  rs1, rs2, rs3 \\
SUB.mp  (rd1, rd2),  rs1, rs2, rs3 \\
ACC.mp  (rd1, rd2),  rs1, rs2, rs3 \\
MAC.mp  (rd1, rd2),  rs1, rs2, rs3 \\
SLL.mp  (rd1, rd2),  rs1, rs2, rs3 \\
SRL.mp  (rd1, rd2),  rs1, rs2, rs3 \\
EQU.mp  rd1,         rs1, rs2, rs3 \\
LTU.mp  rd1,         rs1, rs2, rs3 \\
GTU.mp  rd1,         rs1, rs2, rs3 \\
LBU.cr  rd , imm(rs1)           \\
LHU.cr  rd , imm(rs1)           \\
LW.cr   rd , imm(rs1)           \\
LUI.cr  rd,  imm16              \\
ORI.cr  rd,  imm16              \\
SB.cr   rs2, imm(rs1)           \\
SH.cr   rs2, imm(rs1)           \\
SW.cr   rs2, imm(rs1)           \\
SCATTER rs, base, offsets       \\
GATHER  rd, base, offsets       \\
}

\newpage
\subsection{Instruction Encodings}

The following variations on the standard RISC-V instruction encoding types are
used in this ISE. The varients are kept as close as possible to the orignals,
while still making use of spare bits due to the 4-bit register address encoding.
We also define a new {\em multi-register} or {\em MR} instruction format, which
is used for the multi-precision arithmetic instructions.

\begin{figure}[h]
\centering
\begin{bytefield}[endianness=big]{32}
\bitheader{0-31}               \\
  \BB{7}{f7}
& \BB{5}{rs2  }
& \BB{5}{rs1  }
& \BB{3}{f3}
& \BB{5}{rd   }
& \BB{8}{opcode}
& \BB{8}{\bf R-type} \\

  \BB{12}{imm[11:0]}
& \BB{1}{h}
& \BB{4}{rs1  }
& \BB{3}{f3}
& \BB{1}{b}
& \BB{4}{rd   }
& \BB{8}{opcode}
& \BB{8}{\bf I-type} \\
  
  \BB{7}{imm[11:5]}
& \BB{1}{b}
& \BB{4}{rs1  }
& \BB{1}{h}
& \BB{4}{rs2  }
& \BB{3}{f3}
& \BB{5}{imm[4:0]}
& \BB{8}{opcode}
& \BB{8}{\bf S-type}
\end{bytefield}
\caption{The three main types of encoding used by the ISE.}
\end{figure}

The following tables show the encodings for all instructions in the ISE,
grouped by their layout.

\bigskip

\begin{figure}[h]
\centering \begin{bytefield}[endianness=big]{32} \bitheader{0-31} \\
\BH{12}{imm[11:0]}&\BH{1}{h}& \BH{4}{rs1  }& \BH{3}{f3 }&\BH{1}{b}& \BH{4}{rd   }& \BH{8}{opcode}& \BH{8}{I-type} \\
\BB{12}{imm[11:0]}&\BB{1}{h}& \BB{4}{rs1  }& \BB{3}{ 001 }&\BB{1}{b}& \BB{4}{rd   }& \BB{8}{\encopcode}& \BB{8}{LBU.cr} \\
\BB{12}{imm[11:0]}&\BB{1}{h}& \BB{4}{rs1  }& \BB{3}{ 010 }&\BB{1}{0}& \BB{4}{rd   }& \BB{8}{\encopcode}& \BB{8}{LHU.cr} \\
\BB{12}{imm[11:0]}&\BB{1}{0}& \BB{4}{rs1  }& \BB{3}{ 011 }&\BB{1}{0}& \BB{4}{rd   }& \BB{8}{\encopcode}& \BB{8}{LW.cr } \\
\BB{12}{imm[11:0]}&\BB{1}{0}& \BB{4}{imm4}& \BB{3}{ 010 }&\BB{1}{1}& \BB{4}{rd   }& \BB{8}{\encopcode}& \BB{8}{LUI.cr} \\
\BB{12}{imm[11:0]}&\BB{1}{0}& \BB{4}{imm4}& \BB{3}{ 011 }&\BB{1}{1}& \BB{4}{rd   }& \BB{8}{\encopcode}& \BB{8}{ORI.cr}
\end{bytefield}
\caption{Load instruction encodings. The {\tt h} and {\tt b} bits refer to
which halfword, and which byte of a halfword respectivley the loaded
value is written too. Where these fields are irrelevent, they are tied to
zero. Here {\tt imm4} in the \ASM{LUI.cr} and \ASM{ORI.cr} instructions
represent bits $15..12$ of the $16$ bit immediate they operate with.}
\end{figure}


\begin{figure}[h]
\centering \begin{bytefield}[endianness=big]{32} \bitheader{0-31} \\
\BH{7}{imm[11:5]}&\BH{1}{b}& \BH{4}{rs2  }&\BH{1}{h}& \BH{4}{rs1  }& \BH{3}{f3}& \BH{5}{imm[4:0]}& \BH{8}{opcode}& \BH{8}{S-type} \\
\BB{7}{imm[11:5]}&\BB{1}{b}& \BB{4}{rs2  }&\BB{1}{h}& \BB{4}{rs1  }& \BB{3}{ 100 }& \BB{5}{imm[4:0]}& \BB{8}{\encopcode}& \BB{8}{SB.CR} \\
\BB{7}{imm[11:5]}&\BB{1}{0}& \BB{4}{rs2  }&\BB{1}{h}& \BB{4}{rs1  }& \BB{3}{ 101 }& \BB{5}{imm[4:0]}& \BB{8}{\encopcode}& \BB{8}{SH.CR} \\
\BB{7}{imm[11:5]}&\BB{1}{0}& \BB{4}{rs2  }&\BB{1}{0}& \BB{4}{rs1  }& \BB{3}{ 110 }& \BB{5}{imm[4:0]}& \BB{8}{\encopcode}& \BB{8}{SW.CR}
\end{bytefield}
\caption{Store instruction encodings. The {\tt h} and {\tt b} bits refer to
which halfword, and which byte of a halfword respectivley the value to be
stored is sourced from. Where these fields are irrelevent, they are tied to
zero.}
\end{figure}


\begin{figure}[h]
\centering
\begin{bytefield}[endianness=big]{32}
\bitheader{0-31}               \\
\BH{7}{   f7  }& \BH{5}{rs2}& \BH{5}{rs1}& \BH{3}{f3 }& \BH{5}{rd}& \BH{8}{opcode}& \BH{8}{R-type} \\
\BB{7}{       }& \BB{5}{rs2}& \BB{5}{rs1}& \BB{3}{000}& \BB{5}{rd}& \BB{8}{\encopcode}& \BB{8}{MV2GPR } \\
\BB{7}{       }& \BB{5}{rs2}& \BB{5}{rs1}& \BB{3}{000}& \BB{5}{rd}& \BB{8}{\encopcode}& \BB{8}{MV2COP } \\
\BB{7}{       }& \BB{5}{rs2}& \BB{5}{rs1}& \BB{3}{000}& \BB{5}{rd}& \BB{8}{\encopcode}& \BB{8}{TWID.B } \\
\BB{7}{       }& \BB{5}{rs2}& \BB{5}{rs1}& \BB{3}{000}& \BB{5}{rd}& \BB{8}{\encopcode}& \BB{8}{TWID.HN} \\
\BB{7}{       }& \BB{5}{rs2}& \BB{5}{rs1}& \BB{3}{000}& \BB{5}{rd}& \BB{8}{\encopcode}& \BB{8}{TWID.LN} \\
\BB{7}{       }& \BB{5}{rs2}& \BB{5}{rs1}& \BB{3}{000}& \BB{5}{rd}& \BB{8}{\encopcode}& \BB{8}{ADD.px } \\
\BB{7}{       }& \BB{5}{rs2}& \BB{5}{rs1}& \BB{3}{000}& \BB{5}{rd}& \BB{8}{\encopcode}& \BB{8}{SUB.px } \\
\BB{7}{       }& \BB{5}{rs2}& \BB{5}{rs1}& \BB{3}{000}& \BB{5}{rd}& \BB{8}{\encopcode}& \BB{8}{MUL.px } \\
\BB{7}{       }& \BB{5}{rs2}& \BB{5}{rs1}& \BB{3}{000}& \BB{5}{rd}& \BB{8}{\encopcode}& \BB{8}{SLL.px } \\
\BB{7}{       }& \BB{5}{rs2}& \BB{5}{rs1}& \BB{3}{000}& \BB{5}{rd}& \BB{8}{\encopcode}& \BB{8}{SRL.px } \\
\BB{7}{       }& \BB{5}{rs2}& \BB{5}{rs1}& \BB{3}{000}& \BB{5}{rd}& \BB{8}{\encopcode}& \BB{8}{ROT.px } \\
\BB{7}{       }& \BB{5}{rs2}& \BB{5}{rs1}& \BB{3}{000}& \BB{5}{rd}& \BB{8}{\encopcode}& \BB{8}{XOR.cr } \\
\BB{7}{       }& \BB{5}{rs2}& \BB{5}{rs1}& \BB{3}{000}& \BB{5}{rd}& \BB{8}{\encopcode}& \BB{8}{AND.cr } \\
\BB{7}{       }& \BB{5}{rs2}& \BB{5}{rs1}& \BB{3}{000}& \BB{5}{rd}& \BB{8}{\encopcode}& \BB{8}{OR.cr  } \\
\BB{7}{       }& \BB{5}{rs2}& \BB{5}{rs1}& \BB{3}{000}& \BB{5}{rd}& \BB{8}{\encopcode}& \BB{8}{SCATTER} \\
\BB{7}{       }& \BB{5}{rs2}& \BB{5}{rs1}& \BB{3}{000}& \BB{5}{rd}& \BB{8}{\encopcode}& \BB{8}{GATHER } \\
\end{bytefield}
\caption{Register type instruction encodings.}
\end{figure}


\begin{figure}[h]
\centering
\begin{bytefield}[endianness=big]{32}
\bitheader{0-31}               \\
  \BH{17}{                               }& \BH{3}{f3 }& \BH{5}{     }& \BH{8}{opcode}& \BH{8}{\bf MR-type} \\
  \BB{17}{                               }& \BB{3}{111}& \BB{5}{     }& \BB{8}{\encopcode}& \BB{8}{\bf MR-type} \\
\end{bytefield}
\caption{Multi-precision integer instruction encodings.}
\end{figure}
