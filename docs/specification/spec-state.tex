\subsection{\XCID state}
\label{sec:spec:state}

% =============================================================================

\subsubsection{General-purpose registers}
\label{sec:spec:state:gpr}

The ISE introduces an additional $16$ by $32$-bit register file which
is used exclusively by the instructions in this ISE.
It is to the crypto ISE what the floating-point register file is to the
RISC-V F extension.

Unlike the RISC-V $\GPR$s, the zeroth $\XCR$ is not tied to zero.

The register file addressing is shown in table \ref{tab:state-addr}.

\begin{table}[t]
\centering
\begin{tabular}{|l|l l l l|}
\hline
\multicolumn{1}{|l|}{3:0} & \textbf{31:24} & \textbf{23:16} & \textbf{15:8} & \textbf{7:0} \\ \hline
{\tt c0  } & 3 & 2 & 1 & 0      \\ \hline
{\tt c1  } & 3 & 2 & 1 & 0      \\ \hline
{\tt c2  } & 3 & 2 & 1 & 0      \\ \hline
{\tt c3  } & 3 & 2 & 1 & 0      \\ \hline
{\tt c4  } & 3 & 2 & 1 & 0      \\ \hline
{\tt c5  } & 3 & 2 & 1 & 0      \\ \hline
{\tt c6  } & 3 & 2 & 1 & 0      \\ \hline
{\tt c7  } & 3 & 2 & 1 & 0      \\ \hline
{\tt c8  } & 3 & 2 & 1 & 0      \\ \hline
{\tt c9  } & 3 & 2 & 1 & 0      \\ \hline
{\tt c10 } & 3 & 2 & 1 & 0      \\ \hline
{\tt c11 } & 3 & 2 & 1 & 0      \\ \hline
{\tt c12 } & 3 & 2 & 1 & 0      \\ \hline
{\tt c13 } & 3 & 2 & 1 & 0      \\ \hline
{\tt c14 } & 3 & 2 & 1 & 0      \\ \hline
{\tt c15 } & 3 & 2 & 1 & 0      \\ \hline
\end{tabular}
\caption{This table shows the per-byte layout of the register file.
Note that there is no {\em zero} register in the ISE.
The register with address $0$ will preserve writes.}
\label{tab:state-addr}
\end{table}

\note{
A smart programmer will remember to clear their $\XCR$s of any secret data
before returning execution to some part of the system they did not write.
}

The number of general purpose registers to put into any ISA and ISE is
always contentious.
We chose 16 general purpose registers for this ISE for the following reasons:

\begin{itemize}
\item As demonstrated by ARM, 16 GPRs is plenty for most computation.
Indeed ARM actually has fewer GPRs due to the special nature of some of their
{\em high} registers.
We believe mirroring the base RISC-V ISA and having
32 GPRs would have been overkill for an ISE such as this, even when the
{\tt F} extension adds 32 registers itself.
\item In area-optimised implementations such as micro-controllers, the
total area of a core is often dominated by the register-file.
For embedded applications which do not need all 32 GPRs, one could use the 
{\tt E} extension, and use the area saved by having 16 GPRs to make room
on the die for the crypto ISE.
\item Fewer registers frees up instruction encoding space which would
normally have been needed for register addresses.
\end{itemize}

This is the only extra architectural state required by the ISE.
We followed the RISC-V principles of avoiding implicit or otherwise
hidden state.

% =============================================================================

\subsubsection{Control and status registers}
\label{sec:spec:state:csr}

The Crypto ISE one Control \& Status Register (CSR) to the
standard set. It is used for feature identification and access
control.

\paragraph{Crypto ISE Control Register (mccr)}

The {\tt mccr} register is a non-standard, read/write, M-level CSR
used to identify which parts of the ISE have been implemented,
enable or disable countermeasures, and to control which privilidge
levels can access the ISE features.

The {\tt mccr} occupies address {\tt 0x7FE}, defined in the RISC-V
Privileged ISA Specification as reserved for non-standard,
read/write CSRs.

For more information on the differences between feature sets, see
section \ref{sec:feature-sets}.

\begin{figure}[t]
\centering
\begin{bytefield}[bitwidth=1.6em,endianness=big]{32}
\bitheader{0-31}               \\
\BB{1}{  R}
\BB{1}{ MP}
\BB{1}{ SG}
\BB{1}{p32}
\BB{1}{p16}
\BB{1}{ p8}
\BB{1}{ p4}
\BB{1}{ p2}
\BB{14}{WIRI}
\BB{1}{  s}
\BB{1}{  u}
\BB{1}{ c7}
\BB{1}{ c6}
\BB{1}{ c5}
\BB{1}{ c4}
\BB{1}{ c3}
\BB{1}{ c2}
\BB{1}{ c1}
\BB{1}{ c0}
\end{bytefield}
\captionsetup{singlelinecheck=off}
\caption[x]{\centering Fields of the {\tt mccr} CSR.}
\label{fig:csr-mccr}
\end{figure}

\begin{table}[t]
\centering
\begin{tabular}{l l l}
\toprule
Field& Bits & Description \\\midrule
R    &31&Is the Random Instructions interface is implemented (set) or not (clear). \\
MP   &30&Are the multi-precision arithmetic instructions are implemented (set) or not (clear). \\
SG   &29&Are the {\tt SCATTER} and {\tt GATHER} instructions are implemented (set) or not (clear). \\
p32  &28&Are pack-widths of 32 are supported (set) or not (clear).\\
p16  &27&Are pack-widths of 16 are supported (set) or not (clear).\\
p8   &26&Are pack-widths of  8 are supported (set) or not (clear).\\
p4   &25&Are pack-widths of  4 are supported (set) or not (clear).\\
p2   &24&Are pack-widths of  2 are supported (set) or not (clear).\\
WIRI & 23..10&Writes ignored, reads ignored. \\
s    & 9 & Allow supervisor mode access to the Crypto ISE. \\
u    & 8 & Allow user mode access to the Crypto ISE. \\
c7..c0 & 7..0 & Countermeasure enable bits. \\
\bottomrule
\end{tabular}
\caption{{\tt mccr} register bit field descriptions.}
\end{table}

All writable bits of the {\tt mccr} may be tied to zero based on the
implementation. If no countermeasures are implemented for example, or
user/supervisor mode is not supported, the corresponding bits may be
constants.

\paragraph{Access Control In The mccr}

The {\tt s} and {\tt u} bits of the {\tt mccr} are used to control which
priviledge levels of the RISC-V core can use the Crypto ISE instructions
and state. 

If {\tt s} is clear, neither supervisor mode or user mode may access the
Crypto ISE. If {\tt s} is set, then supervisor mode may access the Crypto
ISE, and user mode access depends on whether the {\tt u} bit is set.

Machine-mode software always has access to the Crypto ISE.

If a Crypto ISE instruction is encountered in a priviledge mode which
does not have access to the ISE, then an {\em illegal instruction exception}
is raised.

Any access to the {\tt mccr} in anything other than machine mode
causes an {\em illegal instruction exception}.

\paragraph{Countermeasure Control In The mccr.}

The {\tt mccr} includes coarse grain control for implementation specific
hardware side channel countermeasures using the {\tt cX} bits.

It is up to the implementer to specify and document exactly what each
countermeasure bit controls. We expect them to be used to implement
switches for controlling things like constant time computation, random
stalling, instruction shuffling and so on.

\paragraph{ABI Standards.}

All of the state added in the Crypto ISE is considered {\em callee save}
for the purposes of the ABI.
That is, if {\tt function1} calls 
{\tt function2}, then {\tt function2} is responsible for saving to the
stack any $\XCR$ registers it needs, and popping them off before returning.

The $\XCR$ registers $0..8$ are considered as function arguments
and/or return values. 
All other $\XCR$ registers are considered temporaries.

Any registers pushed to the stack should be written in ascending order.
That is, as the stack grows downwards, $c1$ should be written (if needed)
to the address following $c0$.

The $\XCR$ registers which are written to the stack during function calls
should be written {\em after} all of the $\GPR$ and floating-point registers
(if the {\tt F} extension is implemented) have been written.

% =============================================================================
