\subsection{\XCID state}
\label{sec:spec:state}

It is important to recognise the overhead, wrt. both 
time (e.g., due to it needing to be context switched) 
and 
space (e.g., due to the logic required),
relating to {\em any} state added to RV32I by \XCID.

However, it also seems reasonable to align this overhead with that of, for
example, an ISE for floating-point arithmetic: through the addition of a
dedicated floating-point register file, 

many 
of the trade-offs involved, e.g., provision of
a) clear separation of duty,
b) additional capacity,
   and
c) additional (specialised) functionality,
are the same, but motivated by cryptographic workloads.

% =============================================================================

\subsubsection{General-purpose register file}
\label{sec:spec:state:gpr}

\XCID 
adds a

additional $16$ by $32$-bit register file which
is used exclusively by the instructions in this ISE.

The ISE introduces an additional $16$ by $32$-bit register file which
is used exclusively by the instructions in this ISE.
It is to the crypto ISE what the floating-point register file is to the
RISC-V F extension.


%\item As demonstrated by ARM, 16 GPRs is plenty for most computation.
%Indeed ARM actually has fewer GPRs due to the special nature of some of their
%{\em high} registers.
%We believe mirroring the base RISC-V ISA and having
%32 GPRs would have been overkill for an ISE such as this, even when the
%{\tt F} extension adds 32 registers itself.
%\item In area-optimised implementations such as micro-controllers, the
%total area of a core is often dominated by the register-file.
%For embedded applications which do not need all 32 GPRs, one could use the 
%{\tt E} extension, and use the area saved by having 16 GPRs to make room
%on the die for the crypto ISE.
%\item Fewer registers frees up instruction encoding space which would
%normally have been needed for register addresses.
%
%
%\item RV32I specifies~\cite[Section 2.1]{SCARV:RV:ISA:I:17} that
%      $
%      \GPR[*][0] = 0 ,
%      $
%      i.e., that the $0$-th general-purpose register is fixed to $0$: all
%      reads from said register yield $0$.  The analagous special-case is 
%      {\em not} true of 
%      $
%      \XCR[*][0] ,
%      $
%      the $0$-th co-processor register.
%
%\item The use of a $\GPR$ register as the base address for memory access,
%      e.g., per \VERB[RV]{xc.ld.w}, \VERB[RV]{xc.st.w}, and similar,
%      aligns with \REFSEC{sec:bg:concept}, where the host core is pitched
%      as a control-path for the co-processor: control-flow orchestration 
%      address computation both fall under the remit of the former, and 
%      can be supported by RV32I as is.
%
%\item Due to the nature of such workloads, however, attack vectors such as
%      LazyFP~\cite{SCARV:StePre:18}, which capitalises on a short-cut wrt. 
%      the overhead of context switching, {\em must} be robustly mitigated.
    
% =============================================================================

\subsubsection{Control and Status Registers (CSRs)}
\label{sec:spec:state:csr}

\begin{table}[p]
\begin{center}
\begin{tabular}{|r@{\;}l|cc|}
\hline
\multicolumn{2}{|c|}{Name}      & Address         & Access     \\
\hline
\XCID Control Register & (XCCR) & \RADIX{7C0}{16} & read/write \\
\XCID Status  Register & (XCSR) & \RADIX{FC0}{16} & read-only  \\
\hline
\end{tabular}
\end{center}
\caption{XXX \cite[Table 2.1]{SCARV:RV:ISA:II:17}}
\label{tab:csr}
\end{table}

\begin{figure}[p]
\begin{center}
\begin{bytefield}[bitwidth={1.4em},bitheight={8.0ex},endianness=big]{32}
\bitheader{0-31}               
\\
  \bitbox{19}{\rule{\width}{\height}}
& \bitbox{ 1}{\rotatebox{90}{\tiny $PACK32$}}
& \bitbox{ 1}{\rotatebox{90}{\tiny $PACK16$}}
& \bitbox{ 1}{\rotatebox{90}{\tiny $PACK8 $}}
& \bitbox{ 1}{\rotatebox{90}{\tiny $PACK4 $}}
& \bitbox{ 1}{\rotatebox{90}{\tiny $PACK2 $}}
& \bitbox{ 2}{\rule{\width}{\height}}
& \bitbox{ 1}{\rotatebox{90}{\tiny $AES   $}}
& \bitbox{ 1}{\rotatebox{90}{\tiny $MP    $}}
& \bitbox{ 1}{\rotatebox{90}{\tiny $PACK  $}}
& \bitbox{ 1}{\rotatebox{90}{\tiny $BIT   $}}
& \bitbox{ 1}{\rotatebox{90}{\tiny $MEM   $}}
& \bitbox{ 1}{\rotatebox{90}{\tiny $RND   $}}
\\
\end{bytefield}
\end{center}
\caption{A diagramatic description of the $\SPR{XCSR}$ register; note that blanked out bits are reserved.}
\label{tab:xcsr:reg}
\end{figure}

\begin{figure}[p]
\begin{center}
\begin{bytefield}[bitwidth={1.4em},bitheight={8.0ex},endianness=big]{32}
\bitheader{0-31}               
\\
  \bitbox{32}{\rule{\width}{\height}}
\\
\end{bytefield}
\end{center}
\caption{A diagramatic description of the $\SPR{XCCR}$ register; note that blanked out bits are reserved.}
\label{tab:xccr:reg}
\end{figure}

\begin{table}[p]
\begin{center}
\begin{tabular}{|lc|l@{\;}l@{\;}l|}
\hline
Field    & Bit(s) & \multicolumn{3}{c|}{Description}                                                             \\
\hline
$RND   $ & $ 0$   & Is  & RND      & feature class, per \REFSEC{sec:bg:feature}, supported (set), or not (clear) \\
$MEM   $ & $ 1$   & Is  & MEM      & feature class, per \REFSEC{sec:bg:feature}, supported (set), or not (clear) \\
$BIT   $ & $ 2$   & Is  & BIT      & feature class, per \REFSEC{sec:bg:feature}, supported (set), or not (clear) \\
$PACK  $ & $ 3$   & Is  & PACK     & feature class, per \REFSEC{sec:bg:feature}, supported (set), or not (clear) \\
$MP    $ & $ 4$   & Is  & MP       & feature class, per \REFSEC{sec:bg:feature}, supported (set), or not (clear) \\
$AES   $ & $ 5$   & Is  & AES      & feature class, per \REFSEC{sec:bg:feature}, supported (set), or not (clear) \\
$PACK2 $ & $ 8$   & Are & $w =  2$ & bit sub-words supported (set) for packed operations, or not (clear)         \\
$PACK4 $ & $ 9$   & Are & $w =  4$ & bit sub-words supported (set) for packed operations, or not (clear)         \\
$PACK8 $ & $10$   & Are & $w =  8$ & bit sub-words supported (set) for packed operations, or not (clear)         \\
$PACK16$ & $11$   & Are & $w = 16$ & bit sub-words supported (set) for packed operations, or not (clear)         \\
$PACK32$ & $12$   & Are & $w = 32$ & bit sub-words supported (set) for packed operations, or not (clear)         \\
\hline
\end{tabular}
\end{center}
\caption{A tabular     description of the $\SPR{XCSR}$ register.}
\label{tab:xcsr:desc}
\end{table}

\begin{table}[p]
\begin{center}
\begin{tabular}{|lc|l@{\;}l@{\;}l|}
\hline
Field    & Bit(s) & \multicolumn{3}{c|}{Description}                                                             \\
\hline

\hline
\end{tabular}
\end{center}
\caption{A tabular     description of the $\SPR{XCCR}$ register.}
\label{tab:xccr:desc}
\end{table}


%\XCID 
%adds two 
%non-standard, machine-class CSRs~\cite[Section 2]{SCARV:RV:ISA:II:17}:
%
%\item The \XCID Control Register (XCCR), denoted $\SPR{XCCR}$,
%
%\item The \XCID Status  Register (XCSR), denoted $\SPR{XCCR}$,
%
%
%
%All writable bits of the {\tt mccr} may be tied to zero based on the
%implementation. If no countermeasures are implemented for example, or
%user/supervisor mode is not supported, the corresponding bits may be
%constants.

%\paragraph{Access Control In The mccr}
%
%The {\tt s} and {\tt u} bits of the {\tt mccr} are used to control which
%priviledge levels of the RISC-V core can use the Crypto ISE instructions
%and state. 
%
%If {\tt s} is clear, neither supervisor mode or user mode may access the
%Crypto ISE. If {\tt s} is set, then supervisor mode may access the Crypto
%ISE, and user mode access depends on whether the {\tt u} bit is set.
%
%Machine-mode software always has access to the Crypto ISE.
%
%If a Crypto ISE instruction is encountered in a priviledge mode which
%does not have access to the ISE, then an {\em illegal instruction exception}
%is raised.
%
%Any access to the {\tt mccr} in anything other than machine mode
%causes an {\em illegal instruction exception}.
%
%\paragraph{ABI Standards.}
%
%All of the state added in the Crypto ISE is considered {\em callee save}
%for the purposes of the ABI.
%That is, if {\tt function1} calls 
%{\tt function2}, then {\tt function2} is responsible for saving to the
%stack any $\XCR$ registers it needs, and popping them off before returning.
%
%The $\XCR$ registers $0..8$ are considered as function arguments
%and/or return values. 
%All other $\XCR$ registers are considered temporaries.
%
%Any registers pushed to the stack should be written in ascending order.
%That is, as the stack grows downwards, $c1$ should be written (if needed)
%to the address following $c0$.
%
%The $\XCR$ registers which are written to the stack during function calls
%should be written {\em after} all of the $\GPR$ and floating-point registers
%(if the {\tt F} extension is implemented) have been written.

% =============================================================================
